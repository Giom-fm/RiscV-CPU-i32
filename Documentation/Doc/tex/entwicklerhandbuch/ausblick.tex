\chapter{Ausblick}

    Diese Arbeit zeigt, dass es mit dem \textit{RISC-V} Befehlssatz möglich
    ist einen einfachen Mikrocontoller zu entwerfen und Programmcode auf diesem
    auszuführen. Durch die freizügige Lizenz des Befehlssatzes, die damit
    verbundene Erweiterbarkeit und die freie Wahl eines \textit{FPGAs}, kann der
    entwickelte Softcore optimal an die Anforderung angepasst werden. 
    Nichtdestotrotz ist der Softcore, der in dieser Arbeit entwickelt wurde,
    aus Komplexitätsgründen und den damit verbunden Zeitaufwand eher simpel
    gehalten, sodass ein Anwendungsfall schwer zu bestimmen ist.
    \\
    Wird der Softcore mit einem Mikrocontroller wie dem \textit{Atmega128}
    verglichen fällt auf, dass es einen großen Unterschied bei der Peripherie gibt.
    Hier fehlen gängige Datenbusse wie \textit{SPI} oder \textit{I2C}.
    Zusätzlich fehlen Analog-Digital-Wandler bzw. Digital-Analog-Wandler, 
    Hardwaretimer sowie externe Interrupts. 
    All dies macht das Zusammenspiel des Softcores mit Sensoren oder Aktoren
    schwierig. Ein nachrüsten dieser Funktionalitäten ist jedoch in \textit{VHDL}
    vergleichsweise einfach umzusetzen und könnte im Rahmen einer zukünftigen Arbeit
    geschehen.
    \\
    Von der Peripherie abgesehen gibt es auch Potential bei der Implementierung
    der Befehlssätze für Multiplikation und fließkommazahl Arithmetik.
    Dies würde den Mehraufwand durch Softwareimplementierungen verringern
    und somit die Performanz des Softcores steigern.
    Eine Anbindung von externem Flash für den Programmspeicher und
    RAM würde für manche Aufgaben nützlich sein, da somit mehr Speicher zu Verfügung
    stehen würde. Die daraus zu Verfügung stehenden internen Speicherblöcke
    könnten daraufhin auch als Cache fungieren und die langsamen Speicherzugriffe beschleunigen.
    Sicherlich wäre damit auch ein Wechsel auf eine Pipelinearchitektur
    verbunden.
    
    

