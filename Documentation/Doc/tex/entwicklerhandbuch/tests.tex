\chapter{Tests}

    \section{Unit Tests}
        Als Unittests werden \textit{Modelsim} Simulationen verwendet.
        Die \textit{Testbench} befindet sich unter
        \textit{RiscV-i32/CPU/Design-Test/testbench}.
        Dabei wird jede einzelne Einheit separat von einander getestet.
        Das Ergebnis ist vordefiniert und wird mit dem Testergebnis verglichen.
        Als hilfe dienen Prozeduren aus dem \textit{test\_util.vhd} Modul im
        selben Verzeichnis. Zu diesen Hilfprozeduren zählt
        \textit{compare\_assert} welches das Testergebnis und das zu erwartende Ergebnis
        vergleicht und eine Rückmeldung über den Test ausgibt. Dabei wird
        \texttt{PASSED} ausgegeben wenn der Test Erfolgreich war und
        \texttt{ASSERT <NACHRICHT> -> expect: [<ERWARTET>], actual: [<ERHALTEN>]}
        falls ein Fehler auftrat. Listing \ref{lab:test-alu} zeigt die Anwendung
        von \textit{compare\_assert}.
        \lstinputlisting[style=vhdl,linerange={39-43},caption={ALU Testbench},label={lab:test-alu}]{../../CPU/Design-Test/testbench/test_alu.vhd}


        \subsection{Rechenwerk (test\_alu.vhd)}
            Die Rechenwerktests, testen die arithmetischen sowie die logischen 
            Operationen.
            \begin{description}
                \item[Erwartetes Ergebnis:] Die ALU verrechnet alle Operanden korrekt miteinander.
                \item[Tatsäzliches Ergebnis:] Die Testfälle verhalten sich wie erwartet.
            \end{description}

        \subsection{Vergleichseinheit (test\_comparator.vhd)}
            Die Vergleichstests, testen die vergleiche von zwei Zahlenwerte wie
            wie sie in Kontrollstrukturen verwendet werden.
            \begin{description}
                \item[Erwartetes Ergebnis:] Der Comparator vergleicht die Werte
                korrekt miteinander. 
                \item[Tatsäzliches Ergebnis:] Die Testfälle verhalten sich wie erwartet.
            \end{description}


        \subsection{Multiplexer (test\_mux\_*.vhd)}
            Die Multiplexertests, testen das korrekte durchreichen der Werte an die
            entsprechenden Einheiten.
            \begin{description}
                \item[Erwartetes Ergebnis:] Die Multiplexer reichen, abhängig von den Steuerleitung, die korrekten
                Werte weiter.
                \item[Tatsäzliches Ergebnis:] Die Testfälle verhalten sich wie erwartet.
            \end{description}


        \subsection{Vergleichseinheit (test\_pc.vhd)}
            Die Befehlszählertests, testen das hochzählen bzw. das setzen
            des Befehlszählers abhängig von den Steuerleitungen.
            \begin{description}
                \item[Erwartetes Ergebnis:] Der Befehlszähler zählt bzw. setzt
                den Befehlszähler korrekt. 
                \item[Tatsäzliches Ergebnis:] Die Testfälle verhalten sich wie erwartet.
            \end{description}

        \subsection{Vergleichseinheit (test\_sign\_extender\_mem.vhd)}
            Die Signextendertests, testen das erweitern der ausgelesenen Daten aus dem Speicher
            \begin{description}
                \item[Erwartetes Ergebnis:] Der Signextender erweitert je nach Steuersignal die
                ausgelesenen Daten auf Wordgröße. 
                \item[Tatsäzliches Ergebnis:] Die Testfälle verhalten sich wie erwartet.
            \end{description}

      



    \section{End-to-End Tests}
        Als End-to-End tests wird C-Code verwendet der über die LEDs das Testergebnis
        ausgibt. Der Quellcode der Test befindet sich unter \textit{Firmware/src/}.

        \subsection{Kontrollstrukturen (counter.c)}\label{lab:counter-test}
            Der Zählertest soll zunächst einfache Kontrollstrukturen testen.
            Dafür wird die LED als Acht-Bit-Zähler verwendet und kontinuierlich hochgezählt.
            Wenn das maximum ($2^8 = 255$) erreicht ist wird der Zähler wieder auf null zurück gesetzt.

            \begin{description}
                \item[Erwartetes Ergebnis:] Die LEDs werden Binär hochgezählt bis
                bis das Maximum erreicht wurde und werden dann zurück gesetzt.
                \item[Tatsäzliches Ergebnis:] Der Testfall verhält sich wie erwartet.
            \end{description}


        \subsection{Shiftoperationen (lightshift.c)}\label{lab:lightshift-test}
            Der Shifttest testet die Shiftoperationen der ALU.
            Dafür wird die LED auf eins gesetzt und nach links geshiftet.
            Ist der Maximalwert erreicht wird nach rechts zurück geshiftet.
            Bei erreichen des Minimalwertes wird wieder von vorne begonnen.

            \begin{description}
                \item[Erwartetes Ergebnis:] Es entsteht ein Lauflicht welches von
                rechts nach links und zurück läuft.
                \item[Tatsäzliches Ergebnis:] Der Testfall verhält sich wie erwartet.
            \end{description}

        \subsection{Multiplikation und Division (mul\_div.c)}
            Der Multiplikations- und Divisionstest testet beide Rechenarten, die
            Aufgrund der fehlenden hardware Unterstützung, durch den Compiler, in
            Software umgesetzt werden. Hierfür wird eine abwandlung des Shifttests
            (Siehe \ref{lab:lightshift-test}) verwendet der das shiften
            durch eine Multiplikation bzw. Division mit zwei ersetzt.

            \begin{description}
                \item[Erwartetes Ergebnis:] Es entsteht ein Lauflicht welches von
                rechts nach links und zurück läuft.
                \item[Tatsäzliches Ergebnis:] Der Testfall verhält sich wie erwartet.
            \end{description}

        \subsection{Unterprogrammsprünge (subroutines.c)}
            Um Unterprogrammsprünge zu testen wird eine Abwandlung des Kontrollstrukturtests
            (Siehe \ref{lab:counter-test}) verwendet. Hinzu kommt eine Routine die
            die Sleep funktionalität in ein Unterprogramm kapselt. Um zusätzlich Rückgabewerte zu testen
            wird das erhöhen des Wertes in einer Funktion abgehandelt.

            \begin{description}
                \item[Erwartetes Ergebnis:] Die LEDs werden Binär hochgezählt bis
                bis das Maximum erreicht wurde und werden dann zurück gesetzt.
                \item[Tatsäzliches Ergebnis:] Der Testfall verhält sich wie erwartet.
            \end{description}

        \subsection{Rekursion (recursion.c)}
            Der Rekursionstest inkrementiert den Wert in einer Routine die
            zusätzlich prüft, ob der Wert unter dem maximalwert liegt.
            Ist dies der Fall wird die selbe Routine erneut aufgerufen.
            Dies geschieht solnage bis der Wert den Maximalwert erreicht.
            Die Endlosschleife setzt den Wert anschließend wieder auf null und 
            ruft die Rekursionsroutine erneut auf.

            \begin{description}
                \item[Erwartetes Ergebnis:] Die LEDs werden Binär hochgezählt bis
                bis das Maximum erreicht wurde und werden dann zurück gesetzt.
                \item[Tatsäzliches Ergebnis:] Der Testfall verhält sich wie erwartet.
            \end{description}

        \subsection{Linker (tobig.c)}
            Der Linkertest testet die Speicherpartitionierung in dem ein zu großes
            Array an Daten angelegt wird.

            \begin{description}
                \item[Erwartetes Ergebnis:] Das Compilieren ist Erfolgreich.
                Das Linken zeigt jedoch einen Fehler an, dass nicht genügend Speicher
                zur Verfügung steht um das Array zu Speichern. 
                \item[Tatsäzliches Ergebnis:] Der Testfall verhält sich wie erwartet.
            \end{description}


