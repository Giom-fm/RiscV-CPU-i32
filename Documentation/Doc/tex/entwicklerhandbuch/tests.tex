\chapter{Tests}

    \section{Unit Tests}

    \section{End-to-End Tests}
        Als End-to-End tests wird C-Code verwendet der über die LEDs das Testergebnis
        ausgibt. Der Quellcode der Test befindet sich unter \textit{Firmware/src/}.

        \subsection{Kontrollstrukturen (counter.c)}\label{lab:counter-test}
            Der Zählertest soll zunächst einfache Kontrollstrukturen testen.
            Dafür wird die LED als Acht-Bit-Zähler verwendet und kontinuierlich hochgezählt.
            Wenn das maximum ($2^8 = 255$) erreicht ist wird der Zähler wieder auf null zurück gesetzt.

            \begin{description}
                \item[Erwartetes Ergebnis:] Die LEDs werden Binär hochgezählt bis
                bis das Maximum erreicht wurde und werden dann zurück gesetzt.
                \item[Tatsäzliches Ergebnis:] Der Testfall verhält sich wie erwartet.
            \end{description}
            
        
        \subsection{Shiftoperationen (lightshift.c)}\label{lab:lightshift-test}
            Der Shifttest testet die Shiftoperationen der ALU.
            Dafür wird die LED auf eins gesetzt und nach links geshiftet.
            Ist der Maximalwert erreicht wird nach rechts zurück geshiftet.
            Bei erreichen des Minimalwertes wird wieder von vorne begonnen.

            \begin{description}
                \item[Erwartetes Ergebnis:] Es entsteht ein Lauflicht welches von
                rechts nach links und zurück läuft .
                \item[Tatsäzliches Ergebnis:] Der Testfall verhält sich wie erwartet.
            \end{description}

        \subsection{Multiplikation und Division (mul\_div.c)}
            Der Multiplikations- und Divisionstest testet beide Rechenarten, die
            Aufgrund der fehlenden hardware Unterstützung, durch den Compiler, in
            Software umgesetzt werden. Hierfür wird eine abwandlung des Shifttests
            (Siehe \ref{lab:lightshift-test}) verwendet der das shiften
            durch eine Multiplikation bzw. Division mit zwei ersetzt.

            \begin{description}
                \item[Erwartetes Ergebnis:] Es entsteht ein Lauflicht welches von
                rechts nach links und zurück läuft .
                \item[Tatsäzliches Ergebnis:] Der Testfall verhält sich wie erwartet.
            \end{description}

        \subsection{Unterprogrammsprünge (subroutines.c)}
            Um Unterprogrammsprünge zu testen wird eine Abwandlung des Kontrollstrukturtests
            (Siehe \ref{lab:counter-test}) verwendet. Hinzu kommt eine Routine die
            die Sleep funktionalität in ein Unterprogramm kapselt. Um zusätzlich Rückgabewerte zu testen
            wird das erhöhen des Wertes in einer Funktion abgehandelt.

            \begin{description}
                \item[Erwartetes Ergebnis:] Die LEDs werden Binär hochgezählt bis
                bis das Maximum erreicht wurde und werden dann zurück gesetzt.
                \item[Tatsäzliches Ergebnis:] Der Testfall verhält sich wie erwartet.
            \end{description}

        \subsection{Rekursion (recursion.c)}
            Der Rekursionstest inkrementiert den Wert in einer Routine die
            zusätzlich prüft, ob der Wert unter dem maximalwert liegt.
            Ist dies der Fall wird die selbe Routine erneut aufgerufen.
            Dies geschieht solnage bis der Wert den Maximalwert erreicht.
            Die Endlosschleife setzt den Wert anschließend wieder auf null und 
            ruft die Rekursionsroutine erneut auf.

            \begin{description}
                \item[Erwartetes Ergebnis:] Die LEDs werden Binär hochgezählt bis
                bis das Maximum erreicht wurde und werden dann zurück gesetzt.
                \item[Tatsäzliches Ergebnis:] Der Testfall verhält sich wie erwartet.
            \end{description}

        \subsection{Linker (tobig.c)}
            Der Linkertest testet die Speicherpartitionierung in dem ein zu großes
            Array an Daten angelegt wird.

            \begin{description}
                \item[Erwartetes Ergebnis:] Das Compilieren ist Erfolgreich.
                Das Linken zeigt jedoch einen Fehler an, dass nicht genügend Speicher
                zur Verfügung steht um das Array zu Speichern. 
                \item[Tatsäzliches Ergebnis:] Der Testfall verhält sich wie erwartet.
            \end{description}


