\chapter{Software}


    Das folgenden Kapitel beschreibt das Zusammenspiel verschiedener Softwarschichten
    um eine \textit{Toolchain} abzubilden die in der Lage ist für den Softcore passenden
    Compilate zu erstellen, Speicherpartitionen zu initialisieren sowie den Bootloader
    zu erstellen und zu linken. 

    \section{Compiler und Linker}

        Der \textit{Compiler} wird benötigt um Programmcode
        in \textit{RISC-V} Maschinenbefehle zu übersetzen.
        Zusätzlich formt dieser die Maschineninstruktionen in ein \textit{ELF}-Dateiformat
        (Siehe \ref{lab:elf}).\\
        Dazu wird die offene und freie
        \textit{GCC (GNU Compiler Collection)} verwendet.
        Das \textit{RISC-V} Team hat hierfür schon vorarbeitet geleistet und bietet
        den kompletten Toolchain Quellcode an.
        Dies ermöglicht das Bauen des \textit{Cross-Compilers} sowie von hilfreichen
        Zusatzprogrammen \cite{riscv-toolchain}.
        
        \subsection{Executable and Linkable Format (ELF)}\label{lab:elf}

    \section{Speicher Partitionierung}


    \section{Bootloader}

        Der Bootloader ist Assembler geschrieben und ist dazu da den
        Softcore zu initialisieren, sodass dieser bereit ist Programmcode auszuführen.
        Dabei wird Hauptsächlich die Stackadresse gesetzt und die \textit{Main}
        Funktion des C-Codes aufgerufen  (Listing \ref{lst:crt0s}).
        Hierbei ist zu beachten, dass der Stack von hinten nach vorne wächst.
        Aus diesem Grund wird als Stackadresse das letzte adressierbare Byte (0xFFFF) benutzt.
        \lstinputlisting[style=cstyle,caption={Bootloader crt0.s},label=lst:crt0s]{../../Firmware/crt0.s}

    \section{Toolchain}
        
        Die Toolchain ist ein \textit{GO}-Programm, namens \textit{Build},
        welches Programmcode compiliert und die \textit{MIFs} (Siehe \ref{lab:mif}).
        Um zu garantieren, dass der Bootloader als erste Instruktion ausgeführt wird,
        wird beim Compilieren der Bootloader durch den Linker eingebunden.
        