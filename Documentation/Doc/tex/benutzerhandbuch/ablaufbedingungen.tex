\section{Ablaufbedingungen}
    Um den Mikrocontroller in Betrieb zu nehmen wird folgeden Soft- und Hardware benötigt.
   

    \subsection{Software}
        Als Betriebsystem wird ein Linux Betriebsystem benötigt.
        \begin{center}
            \begin{longtable}{| l | l | l |}
                \hline
                    Software & Version & Quelle \\
                \hline
                    Quartus Prime Lite & 20.1.1 & \href{https://fpgasoftware.intel.com/?edition=lite}{Intel}\\
                \hline
                    Arrow USB Programmer & 2.4.1-1 & \href{https://wiki.trenz-electronic.de/display/PD/Arrow+USB+Programmer#ArrowUSBProgrammer-DownloadSetupFiles}{Trenz Electronics}\\
                \hline
                    RISC-V GCC Toolchain & 9.2.0 & \href{https://github.com/riscv/riscv-gnu-toolchain}{RISC-V Organisation}\\
                \hline
                \caption{Benötigte Software für Inbetriebnahme}
            \end{longtable}
        \end{center}

    \subsection{Hardware}
        \begin{center}
            \begin{longtable}{| l | l | l |}
                \hline
                    Hardware & Version & Quelle \\
                \hline
                    TEI0003 - CYC1000 & 0.2 & \href{https://wiki.trenz-electronic.de/display/PD/TEI0003+Getting+Started}{Trenz Electorincs}\\
                \hline
                    Micro-USB-Kabel & 2.0 & \\
                \hline
                \caption{Benötigte Hardware für Inbetriebnahme}
            \end{longtable}
        \end{center}