\section{Aufgabenstellung}
    In dem Laborprojekt soll ein Mikrocontroller entwickelt werden der den \textit{RV32I} Befehlssatz implementiert.
    Dieser soll auf ein FPGA board hochgeladen werden können und Programmcode ausführen.
    Zusätzlich soll durch Simulationstests sowie durch Tests durch Programmcode die Korrektheit
    bewiesen werden.

    \subsection{Aufgabenanalyse}
        Die Aufgabenstellung kann in Soft- und Hardware unterteilt werden.

        \subsubsection{Software}
            Aus Softwaresicht wird ein \textit{Compiler} benötigt der in der Lage ist Programmcode
            in \textit{RISC-V} Maschinenbefehle zu übersetzen. Zusätzlich wird ein Dateiformat benötigt
            welches der Mikrocontroller interpretieren kann.
            \\
            Um Programmcode in \textit{RISC-V} Maschinenbefehle zu übersetzen wird die 
            offene und freie \textit{GCC (GNU Compiler Collection)} verwendet.
            Das \textit{RISC-V} Team hat hierfür schon vorarbeitet geleistet und bietet
            den kompletten Toolchain Quellcode an.
            Dies ermöglicht das Bauen des \textit{Cross-Compilers} sowie von hilfreichen Zusatzprogrammen.
            \\
            \url{https://github.com/riscv/riscv-gnu-toolchain}

        \subsubsection{Hardware}
            Aus Hardwaresicht wird ein FPGA-Board benötigt welches eine Möglichkeit bietet
            den in VHDL modellierten Mikrocontroller auf das FPGA-Board sowie Programmcode
            in den Mikrocontroller zu laden.
