\section{Motivation}

    \subsection{RISC-V}
        \textit{RISC-V} ist eine offene und erweiterbare Befehlssatzarchitektur (ISA) die sich an dem
        \textit{RISC (Reduced Instruction Set Computer)} Designprinzip orentiert.
        Dank der freizügigen \textit{BSD-Lizenz} ist es, im gegensatz zu bspw. der \textit{x86} ISA von \textit{Intel},
        jedem erlaubt \textit{RISC-V} Mikroprozessoren zu entwerfen, herzustellen und zu verkaufen.
        Die Lizenz erlaubt es zusätzlich den Befehlssatz nach belieben zu erweitern und somit
        optimal an eine Hardwarearchitektur anzupassen.
        In Tabelle \ref{label:riscv-base} werden die Grundbefehlssätze von RISC-V dargestellt.
        Darüber hinaus besitzt RISC-V noch zusätzliche Befehlssätze die z.B. Hardwaremultiplikation erlauben. \cite{riscv-isa-specs}
        
        \begin{center}
            \begin{longtable}{| l | l | l |}
                \hline
                    Name & Beschreibung & Version \\
                \hline
                    RV32I & 32Bit Basisbefehlssatz mit 32 Registern & 2.1 (Ratifiziert)\\
                \hline
                    RV32E & 32Bit Basisbefehlssatz mit 16 Registern (Eingebettete Systeme) & 1.9 (Offen)\\
                \hline
                    RV64I & 64Bit Basisbefehlssatz & 2.1 (Ratifiziert)\\
                \hline
                    RV128I & 128Bit Basisbefehlssatz & 1.7 (Offen)\\
                \hline
                \caption{RISC-V Grundbefehlssätze}
                \label{label:riscv-base}
            \end{longtable}
        \end{center}

    \subsection{FPGA (Field Programmable Gate Array)}
        Ein FPGA, (Field Programmable Gate Array) ist ein integrierter Schaltkreis (IC) der Digitaltechnik,
        in welchen eine logische Schaltung geladen werden kann.
        Im Vergleich zu der Programmierung von Computern oder Mikrocontrollern wird die Schaltungsstruktur eines FPGAs durch eine
        Hardwarebeschreibungssprache z.B. VHDL beschrieben. Man spricht daher auch von der Konfiguration des FPGA.
        Ohne diese hat der Baustein keine Funktion. \cite{fpga-wiki}
        \\
        Durch einen FPGA ist es somit möglich einen \textit{RISC-V} Befehlssatz in VHDL zu formulieren und auf Hardware zu testen,
        ohne Kosten für eine Fertigung des Chips aufbringen zu müssen.

